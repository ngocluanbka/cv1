%% start of file `template.tex'.
%% Copyright 2006-2013 Xavier Danaux (xdanaux@gmail.com).
%
% This work may be distributed and/or modified under the
% conditions of the LaTeX Project Public License version 1.3c,
% available at http://www.latex-project.org/lppl/.


\documentclass[11pt,a4paper,sans]{moderncv}        % possible options include font size ('10pt', '11pt' and '12pt'), paper size ('a4paper', 'letterpaper', 'a5paper', 'legalpaper', 'executivepaper' and 'landscape') and font family ('sans' and 'roman')

% moderncv themes
\moderncvstyle{casual}                             % style options are 'casual' (default), 'classic', 'oldstyle' and 'banking'
\moderncvcolor{blue}                               % color options 'blue' (default), 'orange', 'green', 'red', 'purple', 'grey' and 'black'
%\renewcommand{\familydefault}{\sfdefault}         % to set the default font; use '\sfdefault' for the default sans serif font, '\rmdefault' for the default roman one, or any tex font name
%\nopagenumbers{}                                  % uncomment to suppress automatic page numbering for CVs longer than one page

% character encoding
\usepackage[utf8]{inputenc}                       % if you are not using xelatex ou lualatex, replace by the encoding you are using
%\usepackage{CJKutf8}                              % if you need to use CJK to typeset your resume in Chinese, Japanese or Korean

% adjust the page margins
\usepackage[scale=0.75]{geometry}
%\setlength{\hintscolumnwidth}{3cm}                % if you want to change the width of the column with the dates
%\setlength{\makecvtitlenamewidth}{10cm}           % for the 'classic' style, if you want to force the width allocated to your name and avoid line breaks. be careful though, the length is normally calculated to avoid any overlap with your personal info; use this at your own typographical risks...

% personal data
\name{Luan}{Bui Ngoc}
\title{Curriculum Vitae}                               % optional, remove / comment the line if not wanted
%\address{street and number}{postcode city}{country}% optional, remove / comment the line if not wanted; the "postcode city" and and "country" arguments can be omitted or provided empty
\phone[mobile]{+84~165~9854~6822}                   % optional, remove / comment the line if not wanted
%\phone[fixed]{+2~(345)~678~901}                    % optional, remove / comment the line if not wanted
%\phone[fax]{+3~(456)~789~012}                      % optional, remove / comment the line if not wanted
\email{ngocluan.bka@gmail.com}                               % optional, remove / comment the line if not wanted
%\homepage{https://github.com/ngocluanbka}                         % optional, remove / comment the line if not wanted
\homepage{https://www.linkedin.com/in/bui-ngoc-luan-70b128112/}
%\extrainfo{additional information}                 % optional, remove / comment the line if not wanted
\photo[74pt][0.2pt]{picture.jpg}                       % optional, remove / comment the line if not wanted; '64pt' is the height the picture must be resized to, 0.4pt is the thickness of the frame around it (put it to 0pt for no frame) and 'picture' is the name of the picture file
\quote{}                                 % optional, remove / comment the line if not wanted

% to show numerical labels in the bibliography (default is to show no labels); only useful if you make citations in your resume
%\makeatletter
%\renewcommand*{\bibliographyitemlabel}{\@biblabel{\arabic{enumiv}}}
%\makeatother
%\renewcommand*{\bibliographyitemlabel}{[\arabic{enumiv}]}% CONSIDER REPLACING THE ABOVE BY THIS

% bibliography with mutiple entries
%\usepackage{multibib}
%\newcites{book,misc}{{Books},{Others}}
%----------------------------------------------------------------------------------
%            content
%----------------------------------------------------------------------------------
\begin{document}
%\begin{CJK*}{UTF8}{gbsn}                          % to typeset your resume in Chinese using CJK
%-----       resume       ---------------------------------------------------------
\makecvtitle

\section{Education}
\cventry{2012--2017}{Bachelor}{Hanoi University of Science and Technology}{Hanoi}{}{}  % arguments 3 to 6 can	 be left empty
\cventry{2009--2012}{Highschool Diploma}{Kim Thanh HighSchool}{Hai Duong}{}{}
%\section{Master thesis}
%\cvitem{title}{\emph{Title}}
%\cvitem{supervisors}{Supervisors}
%\cvitem{description}{Short thesis abstract}

\section{Experience}
\subsection{Vocational}
\cventry{01-2018 03-2017}{Sofware developer}{FOSEC .JSC}{Hanoi}{}{After graduating from HUST with a “Very Good” Degree of Engineering, I got my first full-time job at Fosec as a Python developer.\newline{}%
Detailed achievements:%
\begin{itemize}%
\item Fosec Key
  \begin{itemize}%
	\item My first project in Fosec was “Fosec Key”, which helps many companies manage their servers easily. The main idea of the project is for managers to manage their users accounts, manage their server and access permission between users and servers. 
	\item My role in this project was backend developer, I wrote REST API code and integrated them with 3rd party (privacyIDEA) for bringing 2-factor authentication feature.
  \end{itemize}
\item VSM (Vulnerability management system)
  \begin{itemize}
	\item This project enables to control and synthesize information of multiple vulnerability scanners (Nmap, Acunetix, Nessus) on a single interface.
	\item I worked in this project as a trainee project manager, I analyzed user requests, assigned and supervised work for the team members. 
	\item Beside of this, I had to handle the working flow between VSM and other scanners using Celery Task Queue.
  \end{itemize}
\end{itemize}
}

\cventry{01-2017 10-2015}{Intership}{High Performance Computing Center, HUST}{Hanoi}{}{I have been following my lecturer and joined ICSE as an intern.\newline{}%
Detailed achievements:%
\begin{itemize}%
\item Auto scaling Project
  \begin{itemize}%
	\item This project serves optimization requirement of using resource (RAM, CPU, space).  In this project, my team used cAvisor to collect Docker Container resource data and saved to InfluxDB. After that, our service based on rules would decid to scale up or scale down number of container.
  \end{itemize}
\item IoT Platform Dashboard
  \begin{itemize}
	\item This is my graduate project which solves problem how to integrate and control all IoT Platform in a single interface. I used Python/Flask to write Backend (API, driver for IoT Platform). I also used 	Angular2 to write a simple GUI for end users
  \end{itemize}
\end{itemize}}

\cventry{07-2015 01-2015}{Intership}{Viettel Software}{Hanoi}{}{Viettel Software was my first company, then I respect all the time and knowledge that I learnt when I worked there.
\newline{}  
Detailed achievements:%
  \begin{itemize}%
	\item Trained about Java SE with military working style.
	\item Researched, deployed Apache Hadoop and some Java framework such as JSF, Spring...
  \end{itemize}}
%\subsection{Miscellaneous}
%\cventry{Jan2016}{Web Developer}{}{Hanoi}{}{During the time I learned PHP for my student project, I also make a introduction website using PHP/Laravel}

\section{Adwards and Scholarships}
\cvitem{11-2015}{Third prize in UET Hackathon Competition}
\cvitem{11-2012}{Second prize in the Provincial Level Contest of Physics for High School Students}

\section{References}
\begin{cvcolumns}
  \cvcolumn{Binh Minh Nguyen PhD}{
  \begin{itemize}
  	\item Department of Information Systems School of Information and Communication Technology,Hanoi University of Science and Technology
  	\item Addr: B1 Building, Room 604 Dai Co Viet Rd.,1, Hanoi, Vietnam
  	\item Phone: +84 4 3869 6124
  	\item Mail: minhnb.soict.hust.edu.vn
	\item Web: http://soict.hust.edu.vn/minhnb
  \end{itemize} }
\end{cvcolumns}

% Publications from a BibTeX file without multibib
%  for numerical labels: \renewcommand{\bibliographyitemlabel}{\@biblabel{\arabic{enumiv}}}% CONSIDER MERGING WITH PREAMBLE PART
%  to redefine the heading string ("Publications"): \renewcommand{\refname}{Articles}
%\nocite{*}
\bibliographystyle{plain}
\bibliography{publications}                        % 'publications' is the name of a BibTeX file

% Publications from a BibTeX file using the multibib package
%\section{Publications}
%\nocitebook{book1,book2}
%\bibliographystylebook{plain}
%\bibliographybook{publications}                   % 'publications' is the name of a BibTeX file
%\nocitemisc{misc1,misc2,misc3}
%\bibliographystylemisc{plain}
%\bibliographymisc{publications}                   % 'publications' is the name of a BibTeX file
\end{document}


%% end of file `template.tex'.
