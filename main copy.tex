%% start of file `template.tex'.
%% Copyright 2006-2013 Xavier Danaux (xdanaux@gmail.com).
%
% This work may be distributed and/or modified under the
% conditions of the LaTeX Project Public License version 1.3c,
% available at http://www.latex-project.org/lppl/.


\documentclass[11pt,a4paper,sans]{moderncv}        % possible options include font size ('10pt', '11pt' and '12pt'), paper size ('a4paper', 'letterpaper', 'a5paper', 'legalpaper', 'executivepaper' and 'landscape') and font family ('sans' and 'roman')

% moderncv themes
\moderncvstyle{casual}                             % style options are 'casual' (default), 'classic', 'oldstyle' and 'banking'
\moderncvcolor{blue}                               % color options 'blue' (default), 'orange', 'green', 'red', 'purple', 'grey' and 'black'
%\renewcommand{\familydefault}{\sfdefault}         % to set the default font; use '\sfdefault' for the default sans serif font, '\rmdefault' for the default roman one, or any tex font name
%\nopagenumbers{}                                  % uncomment to suppress automatic page numbering for CVs longer than one page

% character encoding
\usepackage[utf8]{inputenc}                       % if you are not using xelatex ou lualatex, replace by the encoding you are using
%\usepackage{CJKutf8}                              % if you need to use CJK to typeset your resume in Chinese, Japanese or Korean

% adjust the page margins
\usepackage[scale=0.75]{geometry}
%\setlength{\hintscolumnwidth}{3cm}                % if you want to change the width of the column with the dates
%\setlength{\makecvtitlenamewidth}{10cm}           % for the 'classic' style, if you want to force the width allocated to your name and avoid line breaks. be careful though, the length is normally calculated to avoid any overlap with your personal info; use this at your own typographical risks...

% personal data
\name{Luan}{Bui Ngoc}
\title{Curriculum Vitae}                               % optional, remove / comment the line if not wanted
%\address{street and number}{postcode city}{country}% optional, remove / comment the line if not wanted; the "postcode city" and and "country" arguments can be omitted or provided empty
\phone[mobile]{+84~359~856~822}                   % optional, remove / comment the line if not wanted
%\phone[fixed]{+2~(345)~678~901}                    % optional, remove / comment the line if not wanted
%\phone[fax]{+3~(456)~789~012}                      % optional, remove / comment the line if not wanted
\email{ngocluan.bka@gmail.com}                               % optional, remove / comment the line if not wanted
%\homepage{https://github.com/ngocluanbka}                         % optional, remove / comment the line if not wanted
\homepage{https://www.linkedin.com/in/bui-ngoc-luan-70b128112/}
%\extrainfo{additional information}                 % optional, remove / comment the line if not wanted
\photo[74pt][0.2pt]{picture.jpg}                       % optional, remove / comment the line if not wanted; '64pt' is the height the picture must be resized to, 0.4pt is the thickness of the frame around it (put it to 0pt for no frame) and 'picture' is the name of the picture file
%\quote{}                                 % optional, remove / comment the line if not wanted

% to show numerical labels in the bibliography (default is to show no labels); only useful if you make citations in your resume
%\makeatletter
%\renewcommand*{\bibliographyitemlabel}{\@biblabel{\arabic{enumiv}}}
%\makeatother
%\renewcommand*{\bibliographyitemlabel}{[\arabic{enumiv}]}% CONSIDER REPLACING THE ABOVE BY THIS

% bibliography with mutiple entries
%\usepackage{multibib}
%\newcites{book,misc}{{Books},{Others}}
%----------------------------------------------------------------------------------
%            content
%----------------------------------------------------------------------------------
\begin{document}
%\begin{CJK*}{UTF8}{gbsn}                          % to typeset your resume in Chinese using CJK
%-----       resume       ---------------------------------------------------------
\makecvtitle

\section{Education}
\cventry{09-2012 01-2017}{Bachelor}{Hanoi University of Science and Technology}{Hanoi}{Vietnam}{
\begin{itemize}
\item[•] Major: Information System.
\item[•] CPA: 3.2 (Top 3\% performance) - Very Good Degree.
\end{itemize}
} % arguments 3 to 6 can	 be left empty
%\section{Master thesis}
%\cvitem{title}{\emph{Title}}
%\cvitem{supervisors}{Supervisors}
%\cvitem{description}{Short thesis abstract}

\section{Experience}
\cventry{09-2018 03-2018}{Sofware developer}{VNDIRECT}{Hanoi}{}{
\begin{itemize}%
\item Business Process Management
\begin{itemize}
\item[•] Worked on internal projects to develop tools/toolkits for BPM to optimize acceleration of user’s data entry tasks.
\item[•] Collected and visualized data using PowerBI.
\end{itemize}
\end{itemize}}


\cventry{03-2018 03-2017}{Sofware developer}{FOSEC .JSC}{Hanoi}{}{
\begin{itemize}%
\item Fosec Key
  \begin{itemize}%
	\item[•] The main idea of the project is for managers to manage their users accounts, manage their server and access permission between users and servers. 
	\item[•] I wrote REST API code and integrated them with 3rd party (privacyIDEA) for bringing 2-factor authentication feature.
  \end{itemize}
\item VSM (Vulnerability management system)
  \begin{itemize}
	\item[•] This project enables to control and synthesize information of multiple vulnerability scanners (Nmap, Acunetix, Nessus) on a single interface.
	\item[•] I had to handle the working flow between VSM and other scanners using Celery Task Queue.
  \end{itemize}
\end{itemize}}

\cventry{01-2017 10-2015}{Intership}{High Performance Computing Center, HUST}{Hanoi}{}{
\begin{itemize}%
\item Auto scaling Project:
  \begin{itemize}%
	\item[•] Used cAdvior to collect Docker Container resource data and saved to InfluxDB.
	\item[•] Built service based on rules would decide to scale up or scale down number of containers.
  \end{itemize}
\item IoT Platform Dashboard
  \begin{itemize}
	\item[•] Solved problem likes integrate and control all IoT Platform in a single interface.
	\item[•] Used Angular/Python for client-server
  \end{itemize}
\end{itemize}}

\cventry{07-2015 01-2015}{Intership}{Viettel Software}{Hanoi}{}{
  \begin{itemize}%
	\item Trained about Java SE with military working style.
	\item Researched, deployed Apache Hadoop and some Java frameworks such as JSF, Spring...
  \end{itemize}}
%\subsection{Miscellaneous}
%\cventry{Jan2016}{Web Developer}{}{Hanoi}{}{During the time I learned PHP for my student project, I also make a introduction website using PHP/Laravel}

\section{Technical skills}
\cvitem{OS}{Linux, Window}
\cvitem{Language}{Python, Typescript}
\cvitem{Framework}{Flask}
\cvitem{IDE}{Pycharm, Visual Studio Code}
\cvitem{Database}{MySQL, PostgresSQL, MongoDB, Redis}
\cvitem{Tools}{Git, Docker(entry level)}

\section{References}
\begin{cvcolumns}
  \cvcolumn{Binh Minh Nguyen PhD}{
  \begin{itemize}
  	\item Department of Information Systems School of Information and Communication Technology,Hanoi University of Science and Technology
  	\item Addr: B1 Building, Room 604 Dai Co Viet Rd.,1, Hanoi, Vietnam
  	\item Phone: +84 4 3869 6124
  	\item Mail: minhnb.soict.hust.edu.vn
	\item Web: http://soict.hust.edu.vn/minhnb
  \end{itemize} }
\end{cvcolumns}

% Publications from a BibTeX file without multibib
%  for numerical labels: \renewcommand{\bibliographyitemlabel}{\@biblabel{\arabic{enumiv}}}% CONSIDER MERGING WITH PREAMBLE PART
%  to redefine the heading string ("Publications"): \renewcommand{\refname}{Articles}
%\nocite{*}
\bibliographystyle{plain}
\bibliography{publications}                        % 'publications' is the name of a BibTeX file

% Publications from a BibTeX file using the multibib package
%\section{Publications}
%\nocitebook{book1,book2}
%\bibliographystylebook{plain}
%\bibliographybook{publications}                   % 'publications' is the name of a BibTeX file
%\nocitemisc{misc1,misc2,misc3}
%\bibliographystylemisc{plain}
%\bibliographymisc{publications}                   % 'publications' is the name of a BibTeX file
\end{document}


%% end of file `template.tex'.